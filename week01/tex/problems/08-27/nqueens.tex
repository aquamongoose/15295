\documentclass[11pt]{article}

% REF: http://www.artofproblemsolving.com/Wiki/index.php/LaTeX:Packages#fancyhdr
\usepackage{fancyhdr}
\usepackage{hyperref}
\usepackage{amsmath}
\usepackage{amssymb}
\usepackage{amsthm}

\pdfpagewidth 8.5in
\pdfpageheight 11in

\pagestyle{fancy}
\headheight 35pt

\lhead{\textbf{\Large The Presidential Challenge}\\Source: classical}
\chead{}
\rhead{15-295, Fall 2014}
\rfoot{}
\cfoot{}
\lfoot{}

%%

\begin{document}
\section*{Problem}
\noindent CMU's President Suresh challenges SCS's finest programmers to solve the following perplexing conundrum. Given a $w$ by $h$ ($1 \le  w, h \le  8$) chessboard, how many ways are there to place $n$ ($1 \le  n \le  10$) queens such that none are attacking each other? Note that queens can attack horizontally, vertically, and diagonally.
Two arrangements are different if there exists a common square such that there is a queen in one arrangement but not in the other.
\section*{Input}
Three positive integers, $w$, $h$, and $n$.
\section*{Output}
A single positive integer, the answer to the problem.
\section*{Example}
Input:

\begin{verbatim}
2 3 2
\end{verbatim}

\noindent Output:

\begin{verbatim}
2
\end{verbatim}

\section*{Notes}
There are two legal arrangements.
\begin{verbatim}
Q..     ..Q
..Q     Q..
\end{verbatim}
Here, a \texttt{Q} denotes a queen, and a \texttt{.} denotes an empty cell.
\end{document}