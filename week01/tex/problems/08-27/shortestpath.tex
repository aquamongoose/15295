\documentclass[11pt]{article}

% REF: http://www.artofproblemsolving.com/Wiki/index.php/LaTeX:Packages#fancyhdr
\usepackage{fancyhdr}
\usepackage{hyperref}
\usepackage{amsmath}
\usepackage{amssymb}
\usepackage{amsthm}

\pdfpagewidth 8.5in
\pdfpageheight 11in

\pagestyle{fancy}
\headheight 35pt

\lhead{\textbf{\Large Unicycling}\\ Source: classical}
\chead{}
\rhead{15-295, Fall 2014}
\rfoot{}
\cfoot{}
\lfoot{} 

%%

\begin{document}
\section*{Problem}
Bill the freshman is unicycling across CMU's campus to find his purpose in life. The campus consists of $n$ landmarks (numbered from $1$ to $n$) which are connected to each other by $m$ bidirectional paths ($2 \le n \le 10^6, 1 \le  m \le  10^6$). Since Bill is late for ultimate frisbee practice, he would to find a path from landmark $1$ (the Gates building) to landmark $n$ (the Cut) which passes by as few landmarks as possible (including landmarks $1$ and $n$). Please help Bill find the shortest possible length. It is guaranteed that there will always exist at least one path from the Gates building to the Cut.

\section*{Input}
The first line consists of two space-separated positive integers, $n$ and $m$. The subsequent $m$ lines consist of two space-separated positive integers, the indices of two landmarks connected by a bidirectional path.
\section*{Output}
A single positive integer, the number of landmarks visited on the shortest path from landmark $1$ to landmark $n$.
\section*{Example}
Input:

\begin{verbatim}
3 3
1 2
2 3
3 1
\end{verbatim}

\noindent Output:
\begin{verbatim}
2
\end{verbatim}

\section*{Notes}
Bill could take a detour at landmark $2$ (Stever House), but he travels the fastest by going straight to ultimate frisbee practice.

\end{document}