\documentclass[11pt]{article}

% REF: http://www.artofproblemsolving.com/Wiki/index.php/LaTeX:Packages#fancyhdr
\usepackage{fancyhdr}
\usepackage{hyperref}
\usepackage{amsmath}
\usepackage{amssymb}
\usepackage{amsthm}

\pdfpagewidth 8.5in
\pdfpageheight 11in

\pagestyle{fancy}
\headheight 35pt

\lhead{\textbf{\Large Wean Keypad Code}\\ Source: classical}
\chead{}
\rhead{15-295, Fall 2014}
\rfoot{}
\cfoot{}
\lfoot{}

%%

\begin{document}
\section*{Problem}
Josh the freshman has been exploring Wean, and he's run into some trouble.
There's a lock on the door for 21-295, and the only way to open it is to punch the correct number into the keypad.
Earlier in orientation, Josh was given two strings of lowercase English letters, labeled $A$ (of length $N$, such that $1 \le N \le 10^6$) and $B$ (of length $M$, such that $1 \le M \le N$).
Josh knows that to open the door, he will have to find out how many times $B$ appears in $A$.
\linebreak
\linebreak
\noindent More formally, $B$ appears in $A$ starting at position $1 \le i \le N-M+1$ if and only if the substring of $A$ starting at $i$ (that is, the string consisting characters $A_i, A_{i+1}, \hdots, A_{i+N-1}$) is equal to $B$.
The number of times that $B$ appears in $A$ is the number of unique $i$ that exist such that $B$ appears in $A$ starting at position $i$.
\section*{Input}
The first line of input contains the string $A$.
The second line of input contains the string $B$.
\section*{Output}
A single integer, the correct number to get Josh the freshman into 21-295.
\section*{Example}
Input:

\begin{verbatim}
bananana
an
\end{verbatim}

\noindent Output:

\begin{verbatim}
3
\end{verbatim}
\end{document}
