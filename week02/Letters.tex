\documentclass[11pt]{article}

% REF: http://www.artofproblemsolving.com/Wiki/index.php/LaTeX:Packages#fancyhdr
\usepackage{fancyhdr}
%\usepackage{hyperref}
\usepackage{amsmath}
\usepackage{amssymb}
\usepackage{amsthm}

\pdfpagewidth 8.5in
\pdfpageheight 11in

\pagestyle{fancy}
\headheight 35pt

\lhead{\textbf{\Large Letters}\\ Source:
    http://main.edu.pl/en/archive/oi/19/lit\\ Author: Marian M. Kedzierski}
\chead{}
\rhead{Time Limit: 2 seconds\\15-295, Fall 2014}
\rfoot{}
\cfoot{}
\lfoot{}

%%

\begin{document}
\section*{Problem}
Little Johnny has a very long surname. Yet he is not the only such person in his
milieu. As it turns out, one of his friends from kindergarten, Mary, has a
surname of the same length, though different from Johnny's. Moreover, their
surnames contain precisely the same number of letters of each kind - the same
number of letters A, same of letters B, and so on.
\\\\
\noindent Johnny and Mary took to one another and now often play together. One of their
favourite games is to gather a large number of small pieces of paper, write
successive letters of Johnny's surname on them, and then shift them so that they
obtain Mary's surname in the end.
\\\\
\noindent Since Johnny loves puzzles, he has begun to wonder how many swaps of adjacent
letters are necessary to turn his surname into Mary's. For a child his age,
answering such question is a formidable task. Therefore, soon he has asked you,
the most skilled programmer in the kindergarten, to write a program that will
help him.
\section*{Input}
In the first line of the standard input there is a single integer $n$ ($2 \leq n
\leq 10^6$), denoting the length of Johnny's surname. The second line contains
Johnny's surname itself, i.e., contains its $n$ successive letters (without
spaces). The third line contains Mary's surname in the same format: a string of
$n$ letters (with no spaces either). Both strings consist only of capital
(upper-case) letters of the English alphabet.
\section*{Output}
Your program should print a single integer to the standard output: the minimum
number of swaps of adjacent letters that transforms Johnny's surname into
Mary's.
\section*{Example}
Input:

\begin{verbatim}
3
ABC
BCA
\end{verbatim}

\noindent Output:

\begin{verbatim}
2
\end{verbatim}
\end{document}
