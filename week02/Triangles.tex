\documentclass[11pt]{article}

% REF: http://www.artofproblemsolving.com/Wiki/index.php/LaTeX:Packages#fancyhdr
\usepackage{fancyhdr}
%\usepackage{hyperref}
\usepackage{amsmath}
\usepackage{amssymb}
\usepackage{amsthm}

\pdfpagewidth 8.5in
\pdfpageheight 11in

\pagestyle{fancy}
\headheight 35pt

\lhead{\textbf{\Large Triangles}\\ Source: http://main.edu.pl/en/archive/oi/2/tro}
\chead{}
\rhead{15-295, Fall 2014}
\rfoot{}
\cfoot{}
\lfoot{}

%%

\begin{document}
\section*{Problem}
In a finite sequence of positive integers not greater than a billion,
representing lengths of line segments, we want to find three numbers such that
one can build a triangle from segments of such lengths.
\\\\
Write a program which examines whether among the line segments - lengths of
which are written in the standard input - there exist three such that one can
build a triangle from them. If so, the program writes one word TAK ("yes") to
standard output. If there exist no such triple, the program writes one word NIE
("no") to standard output.
\section*{Input}
In the standard input there is a finite sequence of at least three positive
integers not greater than $10^9$, terminated by the number $0$. Each number is
written in a separate line. Positive numbers are lengths of line segments, and
the number $0$ denotes the end of the data.
\section*{Output}
In the standard output there should be either one word NIE, or three lengths of
line segments chosen from the standard input, from which one can build a
triangle. The lengths are separated by single spaces.
\section*{Example 1}
Input:

\begin{verbatim}
105
325
55
12555
1700
0
\end{verbatim}

\noindent Output:

\begin{verbatim}
NIE
\end{verbatim}
\section*{Example 2}
Input:

\begin{verbatim}
250
1
105
150
325
99999
73
0
\end{verbatim}

\noindent Output:

\begin{verbatim}
TAK
\end{verbatim}

\section*{Note}

Note that the input size is REALLY large. Attempting to read in and store all the data will probably result in Time Limit Exceeded or Run Time Error.

\end{document}
